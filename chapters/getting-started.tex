\chapter{Introduction}

\section{Our Notation}

Throughout this tutorial, we'll use the following notation:

\begin{itemize}
\item{} \emph{Italics} is used to denote sketch and feature names, such as \emph{Line 1} and
  \emph{Pulley 1}. While SolidWorks may apply a name like ``Line1@Sketch2'', we'll ignore those names and rename them ourselves.
\item{} \kode{Green code} is used to notate SolidWorks features that you should click, such as \relation{Extrude}, \relation{Circle}, and \relation{Concentric}.
\item{} \texttt{Typewriter text} is used to denote keystrokes, such as \texttt{Tab}
  and \texttt{R}. Here, \texttt{R} refers to pressing the \texttt{R} key alone,
  not in a capitalized form.

\texttt{Shift+R} refers to ``Hold \texttt{Shift} and press \texttt{R}''.
\item{} Constraints will be described as below, which reads ``Add a
    \relation{Coincident}
  relation between \emph{Line 1} and \emph{Line 2}.''
\end{itemize}

\begin{center}
\begin{tabular}{ccc}
  \hline
  \relation{Coincident} & \emph{Line 1} & \emph{Line 2} \\
  \hline
\end{tabular}
\end{center}

\begin{itemize}
  \item{} Removing constraints will be described as below, which reads ``Remove
    the existing \relation{Coincident} relation between \emph{Line 1} and
    \emph{Line 2}.''
\end{itemize}

\begin{center}
\begin{tabular}{ccc}
  \hline
  \xrelation{Coincident} & \emph{\sout{Line 1}} & \emph{\sout{Line 2}} \\
  \hline
\end{tabular}
\end{center}